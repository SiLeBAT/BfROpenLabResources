\documentclass{beamer}

\usepackage[utf8]{inputenc}
\usepackage{hyperref}

\usetheme{Berkeley}
\beamertemplatenavigationsymbolsempty
\setbeamertemplate{headline}{}
 
\title{Clustering in FoodChain-Lab}
\date{}
 
\begin{document}
\maketitle

\section{Task}
\begin{frame}
	\begin{itemize}
		\item Perform a clustering base the following workflow: \url{https://github.com/SiLeBAT/BfROpenLabResources/raw/master/GitHubPages/workflows/Example_Workflow.zip}
		\item Cluster all French stations based on its city.
		\item That means all stations from the same city should be put into one meta-station.
	\end{itemize}
\end{frame}
 
\section{1}
\begin{frame}
	\begin{center}
  		\includegraphics[height=0.6\textheight]{1.png}
	\end{center}
	\begin{itemize}
		\item Import the Example Workflow from \url{https://github.com/SiLeBAT/BfROpenLabResources/raw/master/GitHubPages/workflows/Example_Workflow.zip}.
		\item Open the \textbf{Tracing View} by double-clicking on it.
	\end{itemize}
\end{frame}

\section{2}
\begin{frame}
	\begin{center}
  		\includegraphics[height=0.6\textheight]{2.png}
	\end{center}
	\begin{itemize}
		\item A window showing the delivery network should open now.
	\end{itemize}
\end{frame}

\section{3}
\begin{frame}
	\begin{center}
  		\includegraphics[height=0.6\textheight]{3.png}
	\end{center}
	\begin{itemize}
		\item Right click in the graph to open the context menu and select \textbf{Select Stations}.
	\end{itemize}
\end{frame}

\section{4}
\begin{frame}
	\begin{center}
  		\includegraphics[width=0.9\textwidth]{4.png}
	\end{center}
	\begin{itemize}
		\item You should see this dialog now.
	\end{itemize}
\end{frame}

\section{5}
\begin{frame}
	\begin{center}
  		\includegraphics[width=0.9\textwidth]{5.png}
	\end{center}
	\begin{itemize}
		\item For our clustering we only want to use the French stations, since most primary producers in this data set are French.
		\item Select "Country" as \textbf{Property} and "FR" as \textbf{Value} and press \textbf{OK}.
	\end{itemize}
\end{frame}

\section{6}
\begin{frame}
	\begin{center}
  		\includegraphics[height=0.6\textheight]{6.png}
	\end{center}
	\begin{itemize}
		\item All French stations are seleted now, which is indicated by the blue color.
	\end{itemize}
\end{frame}

\section{7}
\begin{frame}
	\begin{center}
  		\includegraphics[height=0.6\textheight]{7.png}
	\end{center}
	\begin{itemize}
		\item Right click in the graph to open the context menu and select \textbf{Collapse by Property} to cluster the selected stations.
	\end{itemize}
\end{frame}

\section{8}
\begin{frame}
	\begin{center}
  		\includegraphics[width=0.7\textwidth]{8.png}
	\end{center}
	\begin{itemize}
		\item Select \textbf{Yes} to only cluster selected stations.
	\end{itemize}
\end{frame}

\section{9}
\begin{frame}
	\begin{center}
  		\includegraphics[height=0.5\textheight]{9.png}
	\end{center}
	\begin{itemize}
		\item The clustering will be done on city level. That means all stations from the same city will be merged.
		\item Select \textbf{City} and press \textbf{OK}.
	\end{itemize}
\end{frame}

\section{10}
\begin{frame}
	\begin{center}
  		\includegraphics[height=0.5\textheight]{10.png}
	\end{center}
	\begin{itemize}
		\item Just press \textbf{OK}, since we do not want to exclude any cities.
	\end{itemize}
\end{frame}

\section{11}
\begin{frame}
	\begin{center}
  		\includegraphics[height=0.6\textheight]{11.png}
	\end{center}
	\begin{itemize}
		\item All French stations have been clustered to cities.
		\item Each selected station (blue circle) is a French city.
	\end{itemize}
\end{frame}

\section{12}
\begin{frame}
	\begin{center}
  		\includegraphics[height=0.6\textheight]{12.png}
	\end{center}
	\begin{itemize}
		\item Select "PICKING" as \textbf{Editing Mode} and click in the graph to unselect all stations.
		\item You can now see, that some of the stations are yellow. That means, that these stations (French cities) are connected to all outbreak spots (red circles).
	\end{itemize}
\end{frame}

\section{13}
\begin{frame}
	\begin{center}
  		\includegraphics[height=0.6\textheight]{13.png}
	\end{center}
	\begin{itemize}
		\item Since the graph looks confusing now, we should reapply the layout algorithm.
		\item Right click in the graph and select \textbf{Apply Layout $>$ Fruchterman-Reingold} in the context menu.
	\end{itemize}
\end{frame}

\section{14}
\begin{frame}
	\begin{center}
  		\includegraphics[height=0.6\textheight]{14.png}
	\end{center}
	\begin{itemize}
		\item The stations should be arranged in better way now.
		\item The algorithm is not deterministic, therefore your result will look different from the screenshot.		
	\end{itemize}
\end{frame}

\section{15}
\begin{frame}
	\begin{center}
  		\includegraphics[height=0.6\textheight]{15.png}
	\end{center}
	\begin{itemize}
		\item After applying the layout algorithm some stations might be outside the visible area.
		\item To see the whole graph select "TRANSFORMING" as \textbf{Editing Mode} and zoom/move the graph by using the mouse wheel and the left mouse button (works as in Google Maps).
	\end{itemize}
\end{frame}

\end{document}