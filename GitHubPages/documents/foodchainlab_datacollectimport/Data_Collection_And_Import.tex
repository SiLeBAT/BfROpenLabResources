\documentclass[10pt]{beamer}
\usepackage[utf8]{inputenc}
\usepackage{hyperref}
\usepackage[scaled]{helvet}
\usepackage[T1]{fontenc}
\usetheme{Berkeley}
\beamertemplatenavigationsymbolsempty
\setbeamertemplate{headline}{}
\setbeamersize{sidebar width left=1.5cm}
\setbeamerfont{section in sidebar}{size=\fontsize{6}{6}\selectfont}
\setbeamerfont{title in sidebar}{size=\fontsize{6}{6}\selectfont}
\title{Tracing in FoodChain-Lab}
\date{}

\begin{document}
\maketitle

\section{Topics}
\begin{frame}
\leftskip1em\textbf{Learn}
	\begin{itemize}
		\item to fill in a Start Tracing Backward template
    \item to import templates
    \item to load data into the KNIME workflow
	\end{itemize}
\end{frame}

\section{1}
\begin{frame}
	\begin{center}
			\includegraphics[height=0.609cm]{1.png}
	\end{center}
	\begin{itemize}
		\item Open the workflow “FCL_Tracing_Tutorial.knwf“ from the tracing tutorial. \url{https://github.com/SiLeBAT/BfROpenLabResources/raw/master/GitHubPages/workflows/FCL_Tracing_Tutorial.knwf}.
    \item Open the database
	\end{itemize}
\end{frame}

\section{2}
\begin{frame}
\leftskip1em{In Excel:}
	\begin{center}
			\includegraphics[height=0.609cm]{2.png}
	\end{center}
	\begin{itemize}
		\item We need an empty database. Before we reset the database, you can produce a backup: Click the button for backup. A window opens. Choose a folder in which you would like to save your data and click "save".
	\end{itemize}
\end{frame}

\section{3}
\begin{frame}
	\begin{center}
			\includegraphics[height=0.609cm]{3.png}
	\end{center}
	\begin{itemize}
		\item Use the broom to sweep your database clean.
		\item Confirm that you would like to reset the database with „Ja“. The database closes and a new empty database is created
		\item Click „OK“
	\end{itemize}
\end{frame}

\section{4}
\begin{frame}
	\begin{itemize}
		\item Open the empty template „FCL_Backtrace_Start_tob_en.xlsx“ and fill in the data from the delivery papers (see "Dry-Stuff-Inc_START-tracing.docx", \url{https://github.com/SiLeBAT/BfROpenLabResources/raw/master/GitHubPages/documents/foodchainlab_datacollectimport/Dry-Stuff-Inc_START-tracing.docx})
		\item Save the excel template with a different name.
	\end{itemize}
\end{frame}

\section{5a}
\begin{frame}
	\begin{center}
			\includegraphics[height=0.609cm]{5a.png}
	\end{center}
	\begin{itemize}
		\item x
	\end{itemize}
\end{frame}

\section{5b}
\begin{frame}
	\begin{center}
			\includegraphics[height=0.609cm]{5b.png}
	\end{center}
	\begin{itemize}
		\item x
	\begin{center}
			\includegraphics[height=0.609cm]{5c.png}
	\end{center}
	\begin{itemize}
		\item x
	\end{itemize}
\end{frame}

\section{6}
\begin{frame}
	\begin{center}
			\includegraphics[height=0.609cm]{6.png}
	\end{center}
	\begin{itemize}
		\item x
	\end{itemize}
\end{frame}

\end{document}
