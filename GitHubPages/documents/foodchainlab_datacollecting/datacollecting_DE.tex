\documentclass{beamer}

\usepackage[utf8]{inputenc}
\usepackage{hyperref}

\usetheme{Berkeley}
\beamertemplatenavigationsymbolsempty
\setbeamertemplate{headline}{}
 
\title{Erfassen von Daten in FoodChain-Lab}
\date{}
 
\begin{document}
\maketitle

\section{Aufgaben}
\begin{frame}
	\begin{itemize}
		\item Importieren Sie folgende Datei in eine leere Datenbank: \url{https://github.com/SiLeBAT/BfROpenLabResources/raw/master/GitHubPages/documents/1_Lieferliste Caterer 1a.xlsx}
		\item Generieren Sie ein neues Template zur Dateneingabe f\"ur fehlende Daten.
        \item F\"ullen Sie das Template.
        \item Importieren Sie die neuen Daten in die Datenbank.
	\end{itemize}
\end{frame}
 
\section{1}
\begin{frame}
	\begin{center}
  		\includegraphics[height=0.6\textheight]{1.png}
	\end{center}
	\begin{itemize}
		\item \"Offnen Sie das Datenbankfenster über das Men\"u in \textbf{KNIME} via \textbf{Food-Lab} -\textgreater \textbf{Open DB Gui...}.
	\end{itemize}
\end{frame}

\section{2}
\begin{frame}
	\begin{center}
  		\includegraphics[height=0.6\textheight]{2.png}
	\end{center}
	\begin{itemize}
		\item Das Datenbankfenster \"offnet sich.
		\item Dr\"ucken Sie auf das Symbol mit dem Besen in der Toolbar des Fensters. Das bewirkt einen Reset der Datenbank. Achtung: Alle Daten in der lokalen Datenbank werden dadurch gel\"oscht! Eine neue leere Datenbank wird erstellt.
		\item Sie k\"onnen durch Dr\"ucken auf den roten Knopf in der Toolbar vorher eine Sicherung der aktuellen Datenbank erstellen.
	\end{itemize}
\end{frame}

\section{3}
\begin{frame}
	\begin{center}
  		\includegraphics[height=0.3\textheight]{3.png}
	\end{center}
	\begin{itemize}
		\item Bestätigen Sie den Reset der Datenbank mit \textbf{Ja}.
	\end{itemize}
\end{frame}

\section{4}
\begin{frame}
	\begin{center}
  		\includegraphics[width=0.9\textwidth]{4.png}
	\end{center}
	\begin{itemize}
		\item Das Datenbankfenster schließt sich, dann sollten Sie diesen Dialog sehen.
		\item Bestätigen Sie den Dialog mit \textbf{OK}. Dann \"offnet sich das Datenbankfenster wieder mit leeren Tabellen.
	\end{itemize}
\end{frame}

\section{5}
\begin{frame}
	\begin{center}
  		\includegraphics[width=0.9\textwidth]{5.png}
	\end{center}
	\begin{itemize}
		\item Dr\"ucken Sie auf das Ordnersymbol mit dem blauen Pfeil.
		\item Dadurch \"offnet sich der Importdialog.
	\end{itemize}
\end{frame}

\section{6}
\begin{frame}
	\begin{center}
  		\includegraphics[height=0.6\textheight]{6.png}
	\end{center}
	\begin{itemize}
		\item Wählen Sie im Importdialog die Exceldatei aus, die Sie vorher heruntergeladen haben von  \url{https://github.com/SiLeBAT/BfROpenLabResources/raw/master/GitHubPages/documents/1_Lieferliste Caterer 1a.xlsx}
		\item Klicken Sie auf \textbf{\"Offnen}.
	\end{itemize}
\end{frame}

\section{7}
\begin{frame}
	\begin{center}
  		\includegraphics[height=0.6\textheight]{7.png}
	\end{center}
	\begin{itemize}
		\item Die Tabellen der Datenbank haben sich gef\"ullt.
		\item Der erfolgreiche Import wird durch ein kleines Fenster  bekanntgegeben.
		\item Schlie{\ss}en Sie dieses Fenster durch Klicken auf \textbf{OK}.
	\end{itemize}
\end{frame}

\section{8}
\begin{frame}
	\begin{center}
  		\includegraphics[width=0.7\textwidth]{8.png}
	\end{center}
	\begin{itemize}
		\item Dr\"ucken Sie nun auf das Tabellensymbol mit dem gr\"unen Pfeil ganz rechts in der Toolbar.
	\end{itemize}
\end{frame}

\section{9}
\begin{frame}
	\begin{center}
  		\includegraphics[height=0.5\textheight]{9.png}
	\end{center}
	\begin{itemize}
		\item Dadurch \"offnet sich ein Auswahldialog, der es Ihnen erm\"oglicht die Betriebsarten zu definieren, f\"ur die fehlende R\"uckverfolgungsdaten gesammelt werden sollen.
		\item Best\"atigen Sie Ihre Auswahl mit \textbf{OK}.
	\end{itemize}
\end{frame}

\section{10}
\begin{frame}
	\begin{center}
  		\includegraphics[height=0.2\textheight]{10.png}
	\end{center}
	\begin{itemize}
		\item Es werden nun alle fehlenden Daten ermittelt und automatisch Templates generiert, die f\"ur die Datenerfassung optimiert sind.
		\item Im folgenden Fenster wird Ihnen angezeigt, wieviele neue Templates erstellt worden sind und in welchem Ordner sich diese befinden.
		\item Merken Sie sich diesen Ordner bzw. suchen Sie diesen Ordner mit Ihrem Dateimanager auf.
		\item Klicken Sie nun \textbf{OK}.
	\end{itemize}
\end{frame}

\section{11}
\begin{frame}
	\begin{center}
  		\includegraphics[height=0.4\textheight]{11.png}
	\end{center}
	\begin{itemize}
		\item \"Offnen Sie das erzeugte Template durch Doppelklick in Ihrem Dateimanager.
		\item Zur Bearbeitung der Datei ist es neben Excel auch m\"oglich auf freie Software wie z.B. LibreOffice zur\"uckzugreifen.
	\end{itemize}
\end{frame}

\section{12}
\begin{frame}
	\begin{center}
  		\includegraphics[height=0.5\textheight]{12.png}
	\end{center}
	\begin{itemize}
		\item Die Exceldatei besteht aus mehreren Tabellenbl\"attern.
		\item Wesentlich sind: \textbf{Stations}, \textbf{BackTracing}, \textbf{LookUp}
		\item Das Bild zeigt das Tabellenblatt \textbf{BackTracing}, in dem bereits einige Lieferungen automatisch eingetragen worden sind.
		\item Es geht darum die zugeh\"origen Vorlieferungen/Zutaten zu ermitteln.
		\item Im wesentlichen m\"ussen die Felder mit den roten \"Uberschriften ausgef\"ullt/korrigiert werden.
	\end{itemize}
\end{frame}

\section{13}
\begin{frame}
	\begin{center}
  		\includegraphics[height=0.65\textheight]{13.png}
	\end{center}
	\begin{itemize}
		\item Ganz oben, in der 3. Zeile m\"ussen Informationen zum Datenerfasser gemacht werden.
		\item In Zeile 5 steht die Information, in welcher Station die Daten erfasst werden m\"ussen.
		\item In den Zeilen 7-11 befinden sich die Lieferungen, zu denen die konkreten Zutaten gesucht werden.
	\end{itemize}
\end{frame}

\section{14}
\begin{frame}
	\begin{center}
  		\includegraphics[height=0.6\textheight]{14.png}
	\end{center}
	\begin{itemize}
		\item In unserem Fall waren die Information \textbf{Lot size} bereits vorhanden, so da{\ss} hier keine Informationen eingetragen werden m\"ussen. Sollten hier falsche Zahlen stehen m\"ussen sie korrigiert werden.
		\item Es besteht die Möglichkeit eigene \textbf{Additional Fields} zu definieren und Daten dazu erfassen, hier im Beispiel für \textbf{Production Date}.
	\end{itemize}
\end{frame}

\section{15}
\begin{frame}
	\begin{center}
  		\includegraphics[height=0.6\textheight]{15.png}
	\end{center}
	\begin{itemize}
		\item Im Tabellenblatt \textbf{Stations} k\"onnen neue Stationen angelegt werden.
		\item Legen Sie hier die neue Station \textbf{Supermarkt Mueller} an.
		\item Geben Sie der neuen Station den \textbf{Type of business} '\textbf{Zulieferer}'.
	\end{itemize}
\end{frame}

\section{16}
\begin{frame}
	\begin{center}
  		\includegraphics[height=0.6\textheight]{16.png}
	\end{center}
	\begin{itemize}
		\item Die Auswahlm\"oglichkeiten mehrerer Auswahlfelder wie z.B. f\"ur \textbf{Type of business} sind im Tabellenblatt \textbf{LookUp} definiert.
		\item Dort k\"onnen neue LookUps angelegt oder korrigiert werden.
		\item Optimalerweise werden \"Anderungen hierin mit dem zentralen Datenmanager abgesprochen.
	\end{itemize}
\end{frame}

\section{17}
\begin{frame}
	\begin{center}
  		\includegraphics[height=0.6\textheight]{17.png}
	\end{center}
	\begin{itemize}
		\item Im Tabellenblatt \textbf{BackTracing} definieren Sie nun die Zutaten zu den beiden vorgegebenen Lieferungen bzw. Lots.
		\item Dazu f\"ullen Sie die Zeilen ab Zeile 22 aus.
	\end{itemize}
\end{frame}

\section{18}
\begin{frame}
	\begin{center}
  		\includegraphics[height=0.6\textheight]{18.png}
	\end{center}
	\begin{itemize}
		\item Achten Sie darauf zu jeder eingetragenen Lieferung die zugeh\"orige Lotnummer auszuw\"ahlen (Spalte M).
	\end{itemize}
\end{frame}

\section{19}
\begin{frame}
	\begin{center}
  		\includegraphics[height=0.6\textheight]{19.png}
	\end{center}
	\begin{itemize}
		\item Auch hier ist es m\"oglich \textbf{Additional Fields} zu definieren und Daten dazu erfassen, hier im Beispiel für \textbf{Lot fraction [\%]}.
		\item Diese definierten Felder sind dann f\"ur die Analyse genauso verf\"ugbar.
		\item Normalerweise gibt der Datenmanager vor, welche Felder zus\"atzlich ausgef\"ullt werden sollen.
	\end{itemize}
\end{frame}

\section{20}
\begin{frame}
	\begin{center}
  		\includegraphics[height=0.6\textheight]{20.png}
	\end{center}
	\begin{itemize}
		\item Speichern Sie das ausgef\"ullte Template und merken Sie sich den Ordner, in dem Sie es abgespeichert haben.
	\end{itemize}
\end{frame}


\section{21}
\begin{frame}
	\begin{center}
  		\includegraphics[height=0.6\textheight]{21.png}
	\end{center}
	\begin{itemize}
		\item Wechseln Sie wieder zu dem Datenbankfenster, ggf. \"offnen Sie es wieder via \textbf{Food-Lab} -\textgreater \textbf{Open DB Gui...}.
		\item Dr\"ucken Sie auf das Ordnersymbol mit dem blauen Pfeil.
		\item Dadurch \"offnet sich der Importdialog.
		\item Wählen Sie im Importdialog die Exceldatei aus, die Sie soeben abgespeichert haben.
		\item Klicken Sie auf \textbf{\"Offnen}.
	\end{itemize}
\end{frame}

\section{22}
\begin{frame}
	\begin{center}
  		\includegraphics[height=0.3\textheight]{22.png}
	\end{center}
	\begin{itemize}
		\item Die Tabellen der Datenbank haben sich weiter gef\"ullt.
		\item Der erfolgreiche Import wird durch ein kleines Fenster  bekanntgegeben.
		\item Schlie{\ss}en Sie dieses Fenster durch Klicken auf \textbf{OK}.
		\item Sie k\"onnen die Daten jetzt mit FoodChain-Lab analysieren. Schauen Sie dazu bei Bedarf in die entsprechenden Tutorials.
	\end{itemize}
\end{frame}

\end{document}