\documentclass{beamer}

\usepackage[utf8]{inputenc}
\usepackage{hyperref}

\usetheme{Berkeley}
\beamertemplatenavigationsymbolsempty
\setbeamertemplate{headline}{}
 
\title{Importing data into FoodChain-Lab 2}
\date{}
 
\begin{document}
\maketitle

\section{ }

\subsection{Tasks}
\begin{frame}
	\begin{itemize}
		\item In this part of the tutorial we will do a back- and forward-tracing from the "Frozen Fruit Sales" station.
		\item You can either complete part 1 of the tutorial or directly import the following files before starting this tutorial.
		\item Start Template: \url{https://github.com/SiLeBAT/BfROpenLabResources/raw/master/GitHubPages/documents/Start_Tracing_Caterers.xlsx}
		\item Caterer 1 Template: \url{https://github.com/SiLeBAT/BfROpenLabResources/raw/master/GitHubPages/documents/Backtrace_request_Caterer 1.xlsx}
		\item Caterer 2 Template: \url{https://github.com/SiLeBAT/BfROpenLabResources/raw/master/GitHubPages/documents/Backtrace_request_Caterer 2.xlsx}
	\end{itemize}
\end{frame}
 
\subsection{1}
\begin{frame}
	\begin{center}
  		\includegraphics[height=0.6\textheight]{1.png}
	\end{center}
	\begin{itemize}
		\item You should have the database dialog open.
		\item Press the button for generating backtracing templates, which is marked by the red circle.
	\end{itemize}
\end{frame}

\subsection{2}
\begin{frame}
	\begin{center}
  		\includegraphics[width=0.4\textwidth]{2.png}
	\end{center}
	\begin{itemize}
		\item Since we want to do the backtracing for the supplier "Frozen Fruit Sales", select \textbf{Supplier} only and press \textbf{OK}.
	\end{itemize}
\end{frame}

\subsection{3}
\begin{frame}
	\begin{center}
  		\includegraphics[height=0.5\textheight]{3.png}
	\end{center}
	\begin{itemize}
		\item In the file dialog that appears, you can specify the where the generated templates should be saved.
		\item Select the desired folder and press \textbf{Save}.
	\end{itemize}
\end{frame}

\subsection{4}
\begin{frame}
	\begin{center}
  		\includegraphics[width=0.8\textwidth]{4.png}
	\end{center}
	\begin{itemize}
		\item You'll be noticed, that one template was generated in the folder you specified.
		\item Press \textbf{OK}.
	\end{itemize}
\end{frame}

\subsection{5}
\begin{frame}
	\begin{center}
  		\includegraphics[width=0.95\textwidth]{5.png}
	\end{center}
	\begin{itemize}
		\item Open the generated template "Backtrace\_request\_Frozen Fruit Sales.xlsx".
		\item Add the stations from the screenshot to the \textbf{Stations} sheet.
		\item These station include 3 strawberry suppliers that delivered strawberries to "Frozen Fruit Sales" and also a caterer that received strawberries.
	\end{itemize}
\end{frame}

\subsection{6}
\begin{frame}
	\begin{center}
  		\includegraphics[width=0.95\textwidth]{6.png}
	\end{center}
	\begin{itemize}
		\item In the \textbf{BackTracing} sheet you can see the three outgoing deliveries of "Frozen Fruit Sales".
		\item These deliveries belong to lots "108" and "139".
	\end{itemize}
\end{frame}

\subsection{7}
\begin{frame}
	\begin{center}
  		\includegraphics[width=0.95\textwidth]{7.png}
	\end{center}
	\begin{itemize}
		\item Now scroll to the \textbf{Ingredients for Lot(s)} table.
		\item Enter the 3 deliveries, that were used as ingredients for lot "108", from the screenshot.
	\end{itemize}
\end{frame}

\subsection{8}
\begin{frame}
	\begin{center}
  		\includegraphics[width=0.95\textwidth]{8.png}
	\end{center}
	\begin{itemize}
		\item Lot "108" was not only delivered to "Caterer 1" and "Caterer 2", but also to a third caterer.
		\item We can add this information in the \textbf{ForwardTracing\_Opt} sheet.
		\item Enter the delivery to "Caterer 3" from the screenshot.
	\end{itemize}
\end{frame}

\subsection{9}
\begin{frame}
	\begin{center}
  		\includegraphics[height=0.6\textheight]{9.png}
	\end{center}
	\begin{itemize}
		\item To import this file click on the \textbf{Table import} button in the upper left corner of the database dialog.
	\end{itemize}
\end{frame}

\subsection{10}
\begin{frame}
	\begin{center}
  		\includegraphics[height=0.5\textheight]{10.png}
	\end{center}
	\begin{itemize}
		\item In the file dialog that appears, select "Backtrace\_request\_Frozen Fruit Sales.xlsx" and press \textbf{Open}.
	\end{itemize}
\end{frame}

\subsection{11}
\begin{frame}
	\begin{center}
  		\includegraphics[width=0.7\textwidth]{11.png}
	\end{center}
	\begin{itemize}
		\item You'll be notified, that some warnings occurred during import.
		\item Press \textbf{Show Details} to have at look at the warnings.
	\end{itemize}
\end{frame}

\subsection{12}
\begin{frame}
	\begin{center}
  		\includegraphics[width=0.95\textwidth]{12.png}
	\end{center}
	\begin{itemize}
		\item
	\end{itemize}
\end{frame}

\subsection{13}
\begin{frame}
	\begin{center}
  		\includegraphics[height=0.6\textheight]{13.png}
	\end{center}
	\begin{itemize}
		\item
	\end{itemize}
\end{frame}

\subsection{14}
\begin{frame}
	\begin{center}
  		\includegraphics[width=0.95\textwidth]{14.png}
	\end{center}
	\begin{itemize}
		\item
	\end{itemize}
\end{frame}

\subsection{15}
\begin{frame}
	\begin{center}
  		\includegraphics[height=0.5\textheight]{15.png}
	\end{center}
	\begin{itemize}
		\item
	\end{itemize}
\end{frame}

\subsection{16}
\begin{frame}
	\begin{center}
  		\includegraphics[width=0.8\textwidth]{16.png}
	\end{center}
	\begin{itemize}
		\item
	\end{itemize}
\end{frame}

\subsection{17}
\begin{frame}
	\begin{center}
  		\includegraphics[width=0.95\textwidth]{17.png}
	\end{center}
	\begin{itemize}
		\item
	\end{itemize}
\end{frame}

\subsection{18}
\begin{frame}
	\begin{center}
  		\includegraphics[width=0.95\textwidth]{18.png}
	\end{center}
	\begin{itemize}
		\item
	\end{itemize}
\end{frame}

\subsection{19}
\begin{frame}
	\begin{center}
  		\includegraphics[width=0.95\textwidth]{19.png}
	\end{center}
	\begin{itemize}
		\item
	\end{itemize}
\end{frame}

\subsection{20}
\begin{frame}
	\begin{center}
  		\includegraphics[height=0.6\textheight]{20.png}
	\end{center}
	\begin{itemize}
		\item
	\end{itemize}
\end{frame}

\subsection{21}
\begin{frame}
	\begin{center}
  		\includegraphics[height=0.5\textheight]{21.png}
	\end{center}
	\begin{itemize}
		\item
	\end{itemize}
\end{frame}

\subsection{22}
\begin{frame}
	\begin{center}
  		\includegraphics[width=0.4\textwidth]{22.png}
	\end{center}
	\begin{itemize}
		\item
	\end{itemize}
\end{frame}

\end{document}